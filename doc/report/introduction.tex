\section{Introduction}
  The algorithms for summation presented in \cite{repsum} were shown to enjoy the following properties:
  \begin{enumerate}
    \item They compute a reproducible sum independent of the order of the summands, how they are assigned to processors, or how they are aligned in memory.
    \item They make only basic assumptions about the underlying arithmetic (A subset of IEEE Standard 754-2008).
    \item They scale well as a performance-optimized, non-reproducible implementation, as $n$ (number of summands) and $p$ (number of processors) grow.
    \item The user can choose the desired accuracy of the result. In particular, getting a reproducible result with about the same accurace as the performance optimized algorithm is only be a small constant times slower, but higher accuracy is possible too.
  \end{enumerate}
  Our goal is to modify the algorithms in \cite{repsum} so that in addition to the above properties, they enjoy the following properties:
  \begin{enumerate}
    \item The algorithms can be applied in any binary IEEE 754-2008 floating point format, and any valid input in such a format can be summed using the algorithms. The algorithms must be able to sum numbers close to zero, numbers close to overflow, and exceptional values.
    \item They must be expressed as basic operations that can be applied in several applications. They must be able to be built into a user-friendly, performant library.
  \end{enumerate}
